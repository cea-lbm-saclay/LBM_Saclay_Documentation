%% LyX 2.3.6 created this file.  For more info, see http://www.lyx.org/.
%% Do not edit unless you really know what you are doing.
\documentclass[american]{article}
\usepackage[T1]{fontenc}
\usepackage[latin9]{inputenc}
\usepackage{mathrsfs}
\usepackage{amsbsy}
\usepackage{babel}
\begin{document}

\section{Mathematical model}

\section{Phase-field equation with a Lagrange multiplier}

\[
\frac{\partial\phi}{\partial t}+\boldsymbol{\nabla}\cdot(\boldsymbol{u}\phi)=\boldsymbol{\nabla}\cdot\biggl\{ M_{\phi}\left[\boldsymbol{\nabla}\phi-\frac{4}{W_{\phi}}\phi(1-\phi)\boldsymbol{n}_{\phi}+\boldsymbol{\mathscr{L}}(\phi,\varphi,\psi)\right]\biggr\}
\]


\section{Advection of $\psi$}

\[
\frac{\partial\psi}{\partial t}+\boldsymbol{u}_{s}\cdot\boldsymbol{\nabla}\psi=0
\]


\section{Closure}

\[
\varphi=1-\phi-\psi
\]


\section{Def of Lagrange multiplier}

\[
\mathscr{L}(\phi,\varphi,\psi)=\frac{1}{2}\left[\frac{4}{W_{\psi}}\psi(1-\psi)\boldsymbol{n}_{\psi}+\frac{4}{W_{\phi}}\phi(1-\phi)\boldsymbol{n}_{\phi}+\frac{4}{W_{\varphi}}\varphi(1-\varphi)\boldsymbol{n}_{\varphi}\right]
\]


\section{Def of normal}

\[
\boldsymbol{n}_{f}=\frac{\boldsymbol{\nabla}f}{\bigl|\boldsymbol{\nabla}f\bigr|}
\]

where ($f=\psi,\phi,\varphi$)

\section{Navier-Stokes equations with solid phase}

\[
\boldsymbol{\nabla}\cdot\boldsymbol{v}=-\boldsymbol{u}_{s}\cdot\boldsymbol{\nabla}\psi
\]

\[
\varrho\left[\frac{\partial\boldsymbol{v}}{\partial t}+\boldsymbol{\nabla}\cdot(\boldsymbol{v}\boldsymbol{u})\right]=-(1-\psi)\boldsymbol{\nabla}p_{h}+\boldsymbol{\nabla}\cdot\left[\eta\left(\boldsymbol{\nabla}\boldsymbol{v}+(\boldsymbol{\nabla}\boldsymbol{v})^{T}\right)\right]+(1-\psi)\boldsymbol{F}_{tot}
\]

\[
\boldsymbol{v}=(1-\psi)\boldsymbol{u}
\]


\subsection{Interpolations}

\[
\varrho=\phi\rho_{l}+\varphi\rho_{g}+\psi\rho_{s}
\]

\[
\frac{1}{\eta}=\frac{\phi}{\eta_{l}}+\frac{(1-\phi)}{\eta_{g}}
\]


\subsection{Forces}

\[
\boldsymbol{F}_{tot}=\boldsymbol{F}_{c}+\boldsymbol{F}_{g}
\]

\[
\boldsymbol{F}_{c}=\mu_{\phi}\boldsymbol{\nabla}\phi+\mu_{\varphi}\boldsymbol{\nabla}\varphi+\mu_{\psi}\boldsymbol{\nabla}\psi
\]

\[
\boldsymbol{F}_{g}=\varrho(\boldsymbol{\phi})\boldsymbol{g}
\]
where $\boldsymbol{\phi}=(\phi_{0},\phi_{1},\phi_{2})$ and we have
introduced the notations $\phi_{0}\equiv\varphi$, $\phi_{1}\equiv\phi$
and $\phi_{2}\equiv\psi$. With those notations:

\[
\mu_{\phi_{k}}(\boldsymbol{x},t)=\frac{4\gamma_{T}}{W}\sum_{\ell\neq k}\left[\frac{1}{\gamma_{\ell}}\left(\frac{\partial f_{dw}}{\partial\phi_{k}}-\frac{\partial f_{dw}}{\partial\phi_{\ell}}\right)\right]-\frac{3}{4}W\gamma_{k}\boldsymbol{\nabla}^{2}\phi_{k}
\]

\[
\frac{3}{\gamma_{T}}=\sum_{k}\frac{1}{\gamma_{k}}\qquad(e.g.\,\gamma_{1}=\sigma_{10}+\sigma_{12}-\sigma_{20})
\]

\end{document}
