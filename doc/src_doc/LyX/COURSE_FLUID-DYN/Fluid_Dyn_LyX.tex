%% LyX 2.3.6 created this file.  For more info, see http://www.lyx.org/.
%% Do not edit unless you really know what you are doing.
\documentclass[american]{article}
\usepackage[T1]{fontenc}
\usepackage[utf8]{inputenc}
\usepackage{textcomp}
\usepackage{mathrsfs}
\usepackage{amsbsy}
\usepackage{amstext}

\makeatletter

%%%%%%%%%%%%%%%%%%%%%%%%%%%%%% LyX specific LaTeX commands.
%% Because html converters don't know tabularnewline
\providecommand{\tabularnewline}{\\}

\makeatother

\usepackage{babel}
\begin{document}

\section{Fluid dynamics}

The Mach number is defined by

\[
Ma=\frac{\bigl|\boldsymbol{u}\bigr|}{c_{s}}
\]
where $\boldsymbol{u}\equiv\boldsymbol{u}(\boldsymbol{x},t)$ is the
fluid velocity and $c_{s}$ is the sound speed. At 20°C the sound
speed is 344 m/s (1240 km/h) in air and 1500 m/s (5400 km/h) in water.

In the rest of this section, two popular models of Navier-Stokes equations
are detailed for $Ma\ll1$. The first one is the ``low Mach model''
and the second one is the incompressible model.

\subsection{Low Mach Navier-Stokes}

\[
\partial_{t}\rho+\boldsymbol{\nabla}\cdot(\rho\boldsymbol{u})=0
\]

\[
\partial_{t}(\rho\boldsymbol{u})+\boldsymbol{\nabla}\cdot(\rho\boldsymbol{u}\boldsymbol{u})=-\boldsymbol{\nabla}p+\boldsymbol{\nabla}\cdot\left[\eta(\boldsymbol{\nabla}\boldsymbol{u}+(\boldsymbol{\nabla}\boldsymbol{u})^{T})+\left(\eta_{B}-\frac{2}{3}\eta\right)(\boldsymbol{\nabla}\cdot\boldsymbol{u})\boldsymbol{I}\right]+\boldsymbol{F}
\]

where $\rho\equiv\rho(\boldsymbol{x},t)$ is the fluid density, $\boldsymbol{u}\equiv\boldsymbol{u}(\boldsymbol{x},t)$
is the fluid velocity, $p\equiv p(\boldsymbol{x},t)$ is the pressure,
$\eta=\rho\nu$ is the dynamic viscosity and $\nu$ is the kinematic
viscosity, $\eta_{B}$ is the bulk viscosity and $\boldsymbol{I}$
is the identity tensor. In Eq. :eq:`Impulsion\_Balance`, $\boldsymbol{F}$
is the force term and for isothermal fluid, the system is closed by
the Equation of State (EoS)

\[
p=\rho RT_{0}
\]
where $T_{0}$ is the constant temperature and $R$ the specific gas
constant.

\subsection{Incompressible Navier-Stokes}

\[
\boldsymbol{\nabla}\cdot\boldsymbol{u}=0
\]

\[
\rho_{0}\left[\partial_{t}\boldsymbol{u}+\boldsymbol{\nabla}\cdot(\boldsymbol{u}\boldsymbol{u})\right]=-\boldsymbol{\nabla}p_{h}+\boldsymbol{\nabla}\cdot\left[\rho_{0}\nu(\boldsymbol{\nabla}\boldsymbol{u}+(\boldsymbol{\nabla}\boldsymbol{u})^{T})\right]+\boldsymbol{F}
\]

where $p_{h}(\boldsymbol{x},t)$ is the hydrodynamic pressure and
$\rho_{0}$ is the constant density.

\subsection{Boundary Conditions (BC)}

Dirichlet BC

\[
\boldsymbol{u}(\boldsymbol{x}_{b},t)=\boldsymbol{U}_{b}(\boldsymbol{x}_{b},t)
\]

\[
(\boldsymbol{u}-\boldsymbol{U}_{b})\cdot\boldsymbol{n}_{w}=0
\]

\[
(\boldsymbol{u}-\boldsymbol{U}_{b})\cdot\boldsymbol{t}_{w}=0\qquad\text{(no-slip)}
\]

where $\boldsymbol{n}_{w}$ is normal boundary vector and $\boldsymbol{t}_{w}$
tangential boundary vector ($w$: wall)

Neumann BC

\[
\boldsymbol{n}_{w}\cdot\boldsymbol{\sigma}(\boldsymbol{x}_{b},t)=\boldsymbol{T}_{b}(\boldsymbol{x}_{b},t)
\]


\section{Advection-Diffusion equations}

\subsection{Levelset or phase-field equations}

\[
\frac{\partial\phi}{\partial t}+\boldsymbol{\nabla}\cdot(\boldsymbol{u}\phi)=\boldsymbol{\nabla}\cdot\left[M_{\phi}\left(\boldsymbol{\nabla}\phi-\frac{4}{W}\phi(1-\phi)\boldsymbol{n}\right)\right]
\]
where the unknown is the phase-field $\phi\equiv\phi(\boldsymbol{x},t)$
and the inputs are the fluid velocity $\boldsymbol{u}$ and two parameters:
the interface mobility $M_{\phi}$ and the interface width $W$. The
unit normal vector of interface is defined by

\[
\boldsymbol{n}\equiv\boldsymbol{n}(\boldsymbol{x},t)=\frac{\boldsymbol{\nabla}\phi}{\bigl|\boldsymbol{\nabla}\phi\bigr|}
\]


\subsection{Advection-Diffusion Equation (ADE)}

\[
C_{p}\frac{\partial T}{\partial t}+\boldsymbol{\nabla}\cdot(T\boldsymbol{u})=\boldsymbol{\nabla}\cdot(\kappa\boldsymbol{\nabla}T)+\mathscr{S}_{T}
\]

where the unknown is the temperature $T\equiv T(\boldsymbol{x},t)$
and the inputs are the fluid velocity $\boldsymbol{u}$ and two parameters:
the thermal conductivity $\kappa$ and the specific heat $C_{p}$.
A source term $\mathscr{S}_{T}$ can be defined for phase change problems.

\subsection{Boundary conditions}

A generic formulation of bounday conditions for ADE writes:

\[
a_{1}\left.\frac{\partial f}{\partial n}\right|_{\boldsymbol{x}_{b},t}+a_{2}f(\boldsymbol{x}_{b},t)=a_{3}
\]

where $f$ is the differentiable function (here $\phi$ or $T$),
and $a_{1}$, $a_{2}$, $a_{3}$ three scalar values.
\begin{center}
{\small{}}%
\begin{tabular}{llll}
\hline 
\textbf{Name} & $a_{1}$ & $a_{2}$ & $a_{3}$\tabularnewline
\hline 
Dirichlet & 0 & -- & --\tabularnewline
Neumann & -- & 0 & --\tabularnewline
Robin & -- & -- & --\tabularnewline
\hline 
\end{tabular}{\small\par}
\par\end{center}

where ``--'' means non zero value
\end{document}
